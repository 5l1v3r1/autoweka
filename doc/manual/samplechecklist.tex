 
Include in the distribution is a sample experiment definition that runs SMAC on the German Credit dataset from the UCI repository, this section will show you how to run a typical Auto-WEKA experiment.

First, navigate to the \texttt{autoweka} directory that contains the \texttt{autoweka.jar} file. First, we need to build the actual experiment by running the \texttt{ExperimentConstructor}

\begin{aside}
 For all the following Java commands, we assume that you run them inside the \texttt{autoweka} directory, and add both \texttt{autoweka.jar} and \texttt{weka.jar} to the Java's class path.
\end{aside}

\begin{verbatim}
  java -cp autoweka.jar  autoweka.ExperimentConstructor sampleexperiment.xml
\end{verbatim}

This will load the dataset in \texttt{sampledata/creditg.zip}, determine what classifiers and feature selectors that are defined in the \texttt{params} directory can be used, and write out the experiment into the \texttt{experiments} directory.

\begin{aside}
 Note that you will have to modify \texttt{autoweka.smac.SMACExperimentConstructor.properties} to point to your distribution of SMAC, if you are not using the version that came bundled with Auto-WEKA.
\end{aside}

Next, we need to run our experiments by invoking the \texttt{Experiment} with different seeds

\begin{verbatim}
  java -cp autoweka.jar autoweka.Experiment experiments/SMAC-CV10-GermanCredit 0
  java -cp autoweka.jar autoweka.Experiment experiments/SMAC-CV10-GermanCredit 1
  java -cp autoweka.jar autoweka.Experiment experiments/SMAC-CV10-GermanCredit 2
  ...
\end{verbatim}

After watching some paint dry (and the experiments have completed), we now need to generate and combine all the trajectories into a single file using the \texttt{TrajectoryMerger}

\begin{verbatim}
  java -cp autoweka.jar autoweka.TrajectoryParser experiments/SMAC-CV10-GermanCredit 0
  java -cp autoweka.jar autoweka.TrajectoryParser experiments/SMAC-CV10-GermanCredit 1
  java -cp autoweka.jar autoweka.TrajectoryParser experiments/SMAC-CV10-GermanCredit 2
  ...
  java -cp autoweka.jar autoweka.TrajectoryMerger experiments/SMAC-CV10-GermanCredit
\end{verbatim}

\begin{aside}
 If you've defined a number of \texttt{trajectoryPointExtras} in your experiment definition, you'll want to invoke 
 \begin{verbatim}
  java -cp autoweka.jar  autoweka.TrajectoryPointExtraRunner \
              experiments/SMAC-CV10-GermanCredit/SMAC-CV10-GermanCredit.trajectories.0
 \end{verbatim}
 for each completed trajectory before you run the merger.
\end{aside}

Now, you can perform any kind of analysis on this merged trajectory file, but the most common operation you'd want is to find the classifier/hyperparameters that have the best performance using \texttt{GetBestFromTrajectoryGroup}

\begin{verbatim}
  java -cp autoweka.jar  autoweka.tools.GetBestFromTrajectoryGroup \
             experiments/SMAC-CV10-GermanCredit/SMAC-CV10-GermanCredit.trajectory
\end{verbatim}

Additionally, if you use the \classname{autoweka.tools.ExperimentRunner} to execute your experiments, you can use the class \classname{autoweka.tools.TrainedModelPredictionRunner} to make predictions on new data:
\begin{verbatim}
  java -cp autoweka.jar  autoweka.tools.TrainedModelPredictionRunner \
             -model path/to/trained.SEED.model \
             -attributeselection path/to/trained.SEED.attributeselection \
             -dataset path/to/test.arff \
             -predictionpath path/to/predictions.csv
\end{verbatim}
