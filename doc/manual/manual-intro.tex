

Auto-WEKA is a tool that performs combined algorithm selection and hyperparmeter optimisation over the classification and regression algorithms implements in WEKA. More specifically, given a specific dataset, Auto-WEKA explores hyperparameter settings for many algorithms and recommends to a user which method will likely have good generalization performance, using model based optimisation techniques.

%\begin{aside}
% This version of Auto-WEKA is research quality code, and there are still a few steps that are a bit cumbersome.  We are working at making this process much smoother, and any help in fixing these problems is greatly appreciated!
%\end{aside}

%%%%%%%%%%%%%%%%%%%%%%%%%%%%%%%%%%%%%%%%%%%%%%%%%%%%%%%%%%%%%%%%%%%%
\subsection{License}
%%%%%%%%%%%%%%%%%%%%%%%%%%%%%%%%%%%%%%%%%%%%%%%%%%%%%%%%%%%%%%%%%%%%

Auto-WEKA is open source software issued under the \href{http://www.gnu.org/licenses/gpl.html}{GNU General Public License}. Note that each of the optimisation methods that Auto-WEKA have their own license that govern their use.

\subsection{Prerequisites}

Auto-WEKA itself requires only Java 6 or newer to run, while the underlying optimisers that Auto-WEKA uses may have other requirements. Auto-WEKA was developed on Unix-compatible operating systems, but also runs on Windows. Auto-WEKA can use any version of \href{http://www.cs.waikato.ac.nz/ml/weka/}{WEKA}, but it has been targeted against 3.7.9. 

Auto-WEKA makes use of a few modifications to the algorithms in WEKA, detailed inside the \file{autoweka.patch} provided. You can apply the patch to your own WEKA distribution by running \cmdline{patch -Np1 < autoweka.patch} from the WEKA source directory. The changes in this patch just add a support for algorithms to detect if the thread that they have been running in has been interrupted, and then break out of their training phase at their earliest convenience. We have provided a pre-compiled version of WEKA with the patches applied for you to use if you do not wish to compile your own version.

\begin{aside}
Note that you can still use an unmodified version of WEKA, just that you will likely get inferior performance; if Auto-WEKA tries to run a method that takes more than your allotted time budget, it will be equivalent to a method that wasn't able to get a single correct result, while the modified method will report a result that may not be trained to optimality. 
\end{aside}


\subsection{Included Versions}

The Auto-WEKA distribution comes with a ``standalone'' and ``light'' version of Auto-WEKA. The standalone version (found in \file{autoweka.jar}) contains the patched version of WEKA, along with the dependencies that are needed for the GUI. The light version (found in \file{autoweka-light.jar}) \ contains just the Auto-WEKA classes, so you will have to add the WEKA classes (and optionally the classes for the GUI) manually on the class path. Sample scripts that demonstrate how to do this can be found in \file{scripts/autoweka} depending on your platform.

Auto-WEKA makes use of a number of configuration files provided in the \file{params} directory, so you should ensure that you should keep this directory in the same place as either \file{autoweka.jar} or \file{autoweka-light.jar}.
